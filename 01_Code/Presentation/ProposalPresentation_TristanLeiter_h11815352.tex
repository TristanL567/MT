%==== KNITR ===================================================================%



%==== START ===================================================================%

\documentclass[10pt]{beamer}\usepackage[]{graphicx}\usepackage[table]{xcolor}
% maxwidth is the original width if it is less than linewidth
% otherwise use linewidth (to make sure the graphics do not exceed the margin)
\makeatletter
\def\maxwidth{ %
  \ifdim\Gin@nat@width>\linewidth
    \linewidth
  \else
    \Gin@nat@width
  \fi
}
\makeatother

\definecolor{fgcolor}{rgb}{0.345, 0.345, 0.345}
\newcommand{\hlnum}[1]{\textcolor[rgb]{0.686,0.059,0.569}{#1}}%
\newcommand{\hlstr}[1]{\textcolor[rgb]{0.192,0.494,0.8}{#1}}%
\newcommand{\hlcom}[1]{\textcolor[rgb]{0.678,0.584,0.686}{\textit{#1}}}%
\newcommand{\hlopt}[1]{\textcolor[rgb]{0,0,0}{#1}}%
\newcommand{\hlstd}[1]{\textcolor[rgb]{0.345,0.345,0.345}{#1}}%
\newcommand{\hlkwa}[1]{\textcolor[rgb]{0.161,0.373,0.58}{\textbf{#1}}}%
\newcommand{\hlkwb}[1]{\textcolor[rgb]{0.69,0.353,0.396}{#1}}%
\newcommand{\hlkwc}[1]{\textcolor[rgb]{0.333,0.667,0.333}{#1}}%
\newcommand{\hlkwd}[1]{\textcolor[rgb]{0.737,0.353,0.396}{\textbf{#1}}}%
\let\hlipl\hlkwb

\usepackage{framed}
\makeatletter
\newenvironment{kframe}{%
 \def\at@end@of@kframe{}%
 \ifinner\ifhmode%
  \def\at@end@of@kframe{\end{minipage}}%
  \begin{minipage}{\columnwidth}%
 \fi\fi%
 \def\FrameCommand##1{\hskip\@totalleftmargin \hskip-\fboxsep
 \colorbox{shadecolor}{##1}\hskip-\fboxsep
     % There is no \\@totalrightmargin, so:
     \hskip-\linewidth \hskip-\@totalleftmargin \hskip\columnwidth}%
 \MakeFramed {\advance\hsize-\width
   \@totalleftmargin\z@ \linewidth\hsize
   \@setminipage}}%
 {\par\unskip\endMakeFramed%
 \at@end@of@kframe}
\makeatother

\definecolor{shadecolor}{rgb}{.97, .97, .97}
\definecolor{messagecolor}{rgb}{0, 0, 0}
\definecolor{warningcolor}{rgb}{1, 0, 1}
\definecolor{errorcolor}{rgb}{1, 0, 0}
\newenvironment{knitrout}{}{} % an empty environment to be redefined in TeX

\usepackage{alltt}

% --- THEME SETUP ---
\usetheme{Madrid}
\usecolortheme{beaver} % Restores the Red theme

% --- PACKAGES ---
\usepackage[utf8]{inputenc}
\usepackage{amsmath,amssymb}
\usepackage{graphicx}
\usepackage{booktabs}
\usepackage{hyperref}
\usepackage{natbib}
\setcitestyle{authoryear,open={(},close={)}}
\usepackage[table]{xcolor}
\usepackage{caption}

% --- CUSTOM COLOR & FONT OVERRIDES (B) ---
% 1. Define the specific red (Beaver default is usually dark red)
\definecolor{beaverRed}{RGB}{140,45,45} 
\definecolor{MainColour}{RGB}{0,72, 144} 
\definecolor{grey}{RGB}{216,221,227} 
\definecolor{darkgrey}{RGB}{149,153,158} 

% 2. Force Titles to be White on Red
\setbeamercolor{title}{bg=MainColour, fg=white}
\setbeamercolor{frametitle}{bg=MainColour, fg=white}
\setbeamercolor{subtitle}{fg=white} % Subtitle on title page

% 3. Force Footer to be White on Red
\setbeamercolor{date in head/foot}{bg=MainColour, fg=white} 
\setbeamercolor{footline}{bg=MainColour, fg=white}

%%%% Adjusts the text boxes (alterboxes) %%%%
\setbeamercolor{block title alerted}{bg=MainColour, fg=white}
\setbeamercolor{block body alerted}{bg=grey, fg=black}
\setbeamercolor{block title}{bg=darkgrey, fg=white}
\setbeamercolor{block body}{bg=grey, fg=black}

%%%% Enable caption numbering.
\setbeamertemplate{caption}[numbered]

%%%% Customizes the footline / footnoes %%%%
\setbeamertemplate{footline}{%
  \leavevmode%
  \hbox{%
    % LEFT BOX: Date
    \begin{beamercolorbox}[wd=6.4cm,ht=2.25ex,dp=1ex,leftskip=0.5cm]{date in head/foot}%
      \usebeamerfont{date in head/foot}\insertdate
    \end{beamercolorbox}%
    % RIGHT BOX: Page Number
    % 1. REMOVED [right] option to stop it from forcing the text to the edge.
    % 2. ADDED \hfill to push the number to the right side.
    % 3. ADDED \hspace*{...} to create the gap.
    \begin{beamercolorbox}[wd=6.4cm,ht=2.25ex,dp=1ex]{date in head/foot}%
      \usebeamerfont{date in head/foot}\hfill\insertframenumber\hspace*{0.75cm}% <--- Adjust this value
    \end{beamercolorbox}}%
  \vskip0pt%
}

% Remove default navigation symbols
\setbeamertemplate{navigation symbols}{}

% --- METADATA ---
\title[Thesis Proposal]{The Agony and the Ecstasy} 
\subtitle{Constructing a "Crash-Filtered" Equity Index using Machine Learning}
\author{Tristan Leiter}
\institute[WU]{Vienna University of Business and Economics}
\date{January 28, 2026}

%==== Beginning of the document ===============================================%

\IfFileExists{upquote.sty}{\usepackage{upquote}}{}
\begin{document}

%==== Title Page ==============================================================%

% 1. Title Page
\begin{frame}
  \titlepage
\end{frame}

% 2. Outline
% \begin{frame}{Agenda}
%   \tableofcontents
% \end{frame}

%==== 01 - Motivation =========================================================%

\begin{frame}{The "Agony and Ecstasy" of Indexing}
  \begin{itemize}
    \setlength\itemsep{1.2em}
    
    \item \textbf{The Passive Investing Dilemma:} 
    Indices capture the "Ecstasy" of extreme winners but systematically force investors to hold the "Agony" of imploding stocks.
    \begin{itemize}
        \item \citet{Cembalest2014} finds that equity indices are driven by a small tail of winners, while approx. 40\% of constituents suffer "catastrophic declines" ($>70\%$ drawdown) without recovery.
    \end{itemize}

    \item \citet{Tewari2024} formalize this as a \\
    \textbf{Catastrophic Stock Implosion (CSI)}.
    \begin{itemize}
        \item It is a distinct event: a severe price downturn followed by a "zombie" period of prolonged stagnation.
    \end{itemize}

    \item Passive investing captures the winners, but fails to filter the "Agony" until it is too late. Standard metrics fail to distinguish between \textbf{recoverable volatility} and \textbf{terminal implosion}.
  \end{itemize}
\end{frame}

%==== 02 - Research Question ==================================================%

\section{Research Question}
\begin{frame}{Research Question}

\textbf{Main Research Question:}
  \begin{alertblock}{}
    To what extent does a 'Crash-Filtered' equity index, constructed via probabilistic implosion modeling, generate superior risk-adjusted returns compared to the market benchmark and traditional minimum-volatility strategies?
  \end{alertblock}
  
    \vspace{1em}

\textbf{Subquestions:}

  \begin{itemize}
      \setlength\itemsep{0.3em}
    \item Does the integration of latent features derived from Denoising Autoencoders significantly improve the predictive performance (Average Precision) of Ensemble models compared to those trained solely on raw financial data?
    \item Do Ensemble methods exhibit a superior ability to distinguish between recoverable volatility ('Ecstasy') and terminal implosion ('Agony') compared to indiscriminate volatility-based exclusion strategies?
    \item Does shortening the prediction horizon from 12 months to 6 months significantly enhance the index's ability to react to distress signals, or does it introduce excessive turnover without performance gains?
  \end{itemize}
  
  \vspace{1em}

\end{frame}

%==== 03 - Traditional method limitations =====================================%

\begin{frame}{Limits of Traditional Models}
  \begin{itemize}
    \setlength\itemsep{1em}

    \item \textbf{The "Quality Trap":} 
    Perceived safety signals can be misleading. \citet{Penman2018} suggest that low B/P ratios often reflect uncertainty rather than value, while profitability measures lose predictive power over long horizons.

    \item \textbf{The False Positive Dilemma:} 
    Traditional bankruptcy models (e.g., \citet{Altman1968}) and risk scaling strategies, like Minimum-Volatility (MinVol.), fail to distinguish between 'good' volatility (growth) and 'bad' volatility (implosion), systematically excluding winners.

    % \item \textbf{Contribution: A "Crash-Filtered" Index} \\
    % Moving from "volatility forecasting" to "probabilistic implosion modeling":
    % \begin{enumerate}
    %     \item[--] This thesis proposes a Machine-Learning model to predict the "Agony" stocks whilst preserving 
    %     the "Ecstasy" candidates.
    %     \item[--] The model predictions will be used to create a crash-filtered index. Its performance will be 
    %     compared to the market benchmark and traditional MinVol. strategies.
    % \end{enumerate}
    
  \end{itemize}
\end{frame}

%==== 04 - Research Design ====================================================%

%%% Slide 04A - Methodology I:  Dependent Variable %%%

\section{Research Design}
\begin{frame}{Methodology I: Dependent Variable}
    
    \textbf{Goal:} Set up the dependent variable Catastrophic Stock Implosion (CSI)\\
    Following \citet{Tewari2024}, a stock is classified as a CSI ($y=1$) if it satisfies:
    \begin{itemize}
        \item \textbf{Initial Crash (C):} $>80\%$ drawdown  from trailing peak (beginning of the "zombie" period).
        \item \textbf{Non-Recovery:} A maximum cumulative return of -20\% in the "zombie" period.
        \item \textbf{Zombie Period:} Duration of $1.5$ years.
    \end{itemize}
    
    \citet{Tewari2024} employ a yearly prediction horizon ($h=12$ months), which this thesis follows:
    
    \begin{equation}
        y_{i,t} =
        \begin{cases}
        1 & \text{if stock } i \text{ triggers } C = 80\% \text{ within } [t, t+h] \text{ and zombie criteria met}, \\
        0 & \text{otherwise}
        \end{cases}
    \end{equation}

\end{frame}

%%% Slide 04A - Methodology II:  Modelling %%%

\section{Research Design}
\begin{frame}{Methodology II: Modelling}
  \begin{itemize}
    \setlength\itemsep{0.8em}

    \item Using an \textbf{Autoencoder} to compress the noisy financial data into latent "distress structures" that linear PCA misses.
    \item \textbf{Supervised Learning:}
    Training ensemble-methods (Random Forest, XGBoost, CatBoost, LightGBM) on both raw data and latent features to predict the probability of CSI.

  \end{itemize}
  
  \vspace{0.1em}

  \textbf{Cross-Validation} optimization will be conducted via \textbf{Average Precision (AP)}.
  Since the dataset is imbalanced, AP provides a more robust signal for hyperparameter
  tuning without committing to a specific decision threshold. \\
  
  \vspace{0.4em}

  % \textbf{AUC-ROC} might lead to a FPR that is too high even if the Agony stocks are correctly
  % classified. In this case, we would miss out on too many Ecstasy Stocks (e.g. NVIDIA) even though
  % our AUC-ROC Score is high. \\
  
  The models will be evaluated based on \textbf{Recall at fixed FPR}. \\ 
  A constraint-based metric aligns more closely with the practical "Risk-Budget" in Portfolio Management. 
  The objective is to maximize the number of Agony stocks identified (Recall) whilst capping the 
  exclusion of Ecstasy stocks.
  
\end{frame}


%%% Slide 04A - Methodology III:  Index Construction %%%

\section{Research Design}
\begin{frame}{Methodology III: Index Construction}
  
  To ensure a proper \textbf{Out-of-Sample test}, the dataset will be split into three parts:
  
  \begin{itemize}
    \item Training-Set for Cross-Validation
    \item Test-set for Model-selection
    \item Out-of-Sample set for backtesting
  \end{itemize}
  
  The best-performing model within the test-set will be used to predict a CSI for each firm in the consecutive year.
  All firms exceeding the the probability threshold, $\theta$, will be removed.
  In this context, $\theta$ is selected based on the desired FPR rate (for example $3\%$ or $5\%$).
  Additionally, the "Crash-Filtered" Index will be rebalanced anually at the end of each calendar year.

  \vspace{0.3em}

  \begin{block}{Index Construction}
 The "Crash-Filtered" Index systematically excludes constituents where the predicted probability $\hat{p}_{CSI} > \theta$.
Unlike standard classifiers that use a default $\theta=0.5$, this threshold is \textbf{dynamically calibrated} to satisfy a
specific risk constraint (e.g., $FPR \leq 5\%$), ensuring the exclusion rate aligns with the investor's "budget" for opportunity cost.
  \end{block}
  
\end{frame}


%==== 05 - Data ===============================================================%

%%% Slide 05A - CRSP %%%
\section{CRSP}
\begin{frame}{CRSP}

      \textbf{The Universe (CRSP)}
      \begin{itemize}
        \item \textbf{Scope:} US Common Equities (NYSE, AMEX, NASDAQ).
        \item \textbf{Timeline:} 1998 - 2024.
        \item \textbf{Size:} $3,263$ unique firms, 51,773 firm-year observations.
        \item \textbf{Constraint:} Minimum listing lifetime of 5 years to ensure sufficient learning history 
        (13 years on average per firm).
        \item \textbf{CSI-Events:} $2,236$ CSIs in Total. 
      \end{itemize}

\begin{table}[!h]
\centering
\fontsize{7pt}{9pt}\selectfont 
% -------------------------
\begin{tabular}[t]{lrr}
\toprule
\multicolumn{1}{c}{Category} & \multicolumn{2}{c}{Cohort Distribution} \\
\cmidrule(l{2pt}r{3pt}){2-3}
& Imploded Firms & Never Imploded\\
\midrule
\cellcolor{gray!10}{High Growth ($> 10\%$)} & \cellcolor{gray!10}{4.9\%} & \cellcolor{gray!10}{40.5\%}\\
Moderate Growth ($5 - 10\%$) & 9.1\% & 25.9\%\\
\cellcolor{gray!10}{Low Growth ($2 - 5\%$)} & \cellcolor{gray!10}{8.2\%} & \cellcolor{gray!10}{9.0\%}\\
Stagnation ($-2 - 2\%$) & 0.0\% & 0.1\%\\
\cellcolor{gray!10}{Value Destruction (temp. Recovery)} & \cellcolor{gray!10}{64.8\%} & \cellcolor{gray!10}{15.4\%}\\
\addlinespace
Value Destruction (No Recovery) & 2.2\% & 0.8\%\\
\cellcolor{gray!10}{Unknown} & \cellcolor{gray!10}{10.8\%} & \cellcolor{gray!10}{8.4\%}\\
\bottomrule
\multicolumn{3}{l}{\scriptsize \textit{Note:} Categories are measured by the average geometric return.} \\
\end{tabular}
% --------------------------
\captionsetup{font=tiny}
\caption{Categorization of CSI events dependent on event type.}
% --------------------------
\end{table}

\end{frame}

%%% Slide 05B - Compustat %%%

\section{Compustat}
\begin{frame}{Compustat}

      \textbf{Features}
      \begin{itemize}
        \item \textbf{Accounting:} Balance-Sheet and Income-Statement variables.
        \item \textbf{Macro:} Variables with possible interactions with accounting variables (interest rate, others). 
        Obtained from the FRED database.
        \item \textbf{Other:} Specifically targeting "Zombie" precursors:
        \begin{itemize}
            \item \textit{Employees} (Number of employees).
            \item \textit{Rental expenses} 
        \end{itemize}
      \end{itemize}
 
\end{frame}

%%% 05C - Feature table %%%%%%%%%%

\section{Compustat}
\begin{frame}{Compustat}

\begin{table}[ht]
\centering
\fontsize{3pt}{4pt}\selectfont 
% -------------------------
% \caption{Selected Fundamental Features (Compustat)}
\label{tab:compustat_vars}
\scriptsize
\begin{tabular}{llp{4cm}}
\toprule
\textbf{Category} & \textbf{Variable Code} & \textbf{Description} \\
\midrule
\textbf{Balance Sheet: Assets} 
 & \texttt{at} / \texttt{act} & Total Assets / Total Current Assets \\
 & \texttt{che} / \texttt{ivst} & Cash \& Short-Term Inv. / Short-Term Investments \\
 & \texttt{rect} / \texttt{invt} & Receivables (Channel stuffing risk) / Inventories \\
 & \texttt{wcap} & Working Capital (Liquidity buffer) \\
 & \texttt{ppent} / \texttt{intan} & Net PP\&E / Intangibles (Soft assets) \\
 & \texttt{gdwl} & Goodwill (Impairment risk) \\
 & \texttt{txdba} & Deferred Tax Asset (Long Term) - \textit{Proxy for NOLs} \\
\addlinespace
\textbf{Balance Sheet: Liab/Eq} 
 & \texttt{lt} / \texttt{lct} & Total Liabilities / Total Current Liabilities \\
 & \texttt{dltt} / \texttt{dlc} & Long-Term Debt / Debt in Current Liabilities \\
 & \texttt{dd1} & Long-Term Debt Due in 1 Year (\textit{Refinancing wall}) \\
 & \texttt{ap} / \texttt{txp} & Accounts Payable / Income Taxes Payable \\
 & \texttt{txditc} & Deferred Taxes \& Inv. Tax Credit (Non-current) \\
 & \texttt{seq} / \texttt{re} & Stockholders' Equity / Retained Earnings (\textit{Accum. Deficit}) \\
 & \texttt{pstk} / \texttt{mib} & Preferred Stock / Noncontrolling Interest \\
 % & \texttt{tstk} & Treasury Stock (Contra-equity) \\
% \addlinespace
% \textbf{Income Statement} 
%  & \texttt{sale} / \texttt{gp} & Net Sales / Gross Profit \\
%  & \texttt{cogs} / \texttt{xsga} & Cost of Goods Sold / SG\&A Expense \\
%  & \texttt{ebit} / \texttt{oibdp} & EBIT / Operating Income Before Deprec. (\textit{EBITDA}) \\
%  & \texttt{ni} / \texttt{pi} & Net Income / Pretax Income (\textit{Tax efficiency check}) \\
%  & \texttt{epsfi} & Earnings Per Share (Basic) \\
%  & \texttt{spi} & Special Items (\textit{Early warning signal}) \\
%  & \texttt{xint} / \texttt{xinst} & Total Interest Exp. / Short-Term Interest Exp. \\
%  & \texttt{xrd} / \texttt{xad} & R\&D Exp. / Advertising Exp. (\textit{Discretionary cuts}) \\
%  & \texttt{xrent} & Rental Expense (Operational rigidity) \\
%  & \texttt{dp} & Depreciation \& Amortization \\
% \addlinespace
% \textbf{Supplementary} 
%  & \texttt{emp} & Number of Employees (\textit{Real economy anchor}) \\
%  & \texttt{dpr} / \texttt{ppegt} & Accum. Depreciation / Gross PPE (\textit{Asset age ratio}) \\
%  & \texttt{fca} & Foreign Exchange Income/Loss \\
%  & \texttt{mkvalt} & Market Value (Fiscal Year-End) \\
\bottomrule
\end{tabular}
% --------------------------
\captionsetup{font=tiny}
\caption{Partial table of accounting variables.}
% --------------------------
\end{table}

\end{frame}

%==== 06 - Expected Contribution ==============================================%

\section{Expected Contribution}
\begin{frame}{Expected Contribution}

  \begin{itemize}
    \setlength\itemsep{0.6em}

        \item \textbf{Replication:} Confirming the methodology and results of the working paper by
        \citet{Tewari2024} with the CRSP and Compustat Data.

        \item \textbf{Extension:} Extending the results of \citet{Tewari2024} by including autoencoders for extraction
        of the informational content in the features and by using the model-predictions for index-constructions.

        \item \textbf{Challenging traditional risk-scaling approaches:} \\
        Showing that ML-applications are better suited to distinguish between Ecstasy and Agony stocks compared to
        classical risk-scaling methods, like Low-Volatility, Low-Beta or Altman's Z-score. Autoencoders and ensemble-methods
        are well suited for capturing the complex interactions between accounting and macro variables.

    \end{itemize}

\end{frame}

%==== 07 - Appendix ===========================================================%

%%% Slide 07A - Implosion Count per Number of CSI-events per event type %%%
\section{Appendix}
\begin{frame}{Appendix: Number of CSIs per Type}

\begin{table}[!h]
\centering
% \caption{Distribution of Catastrophic Stock Implosion (CSI) Events per Firm}
\centering
\begin{tabular}[t]{crr}
\toprule
Total CSI Events & Number of Firms & Percentage (\%)\\
\midrule
\cellcolor{gray!10}{0} & \cellcolor{gray!10}{2167} & \cellcolor{gray!10}{66.41}\\
1 & 537 & 16.46\\
\cellcolor{gray!10}{2} & \cellcolor{gray!10}{255} & \cellcolor{gray!10}{7.81}\\
3 & 135 & 4.14\\
\cellcolor{gray!10}{4} & \cellcolor{gray!10}{93} & \cellcolor{gray!10}{2.85}\\
\addlinespace
5 & 51 & 1.56\\
\cellcolor{gray!10}{6} & \cellcolor{gray!10}{18} & \cellcolor{gray!10}{0.55}\\
7 & 7 & 0.21\\
\bottomrule
\end{tabular}
% --------------------------
\captionsetup{font=tiny}
\caption{Number of observations per CSI count.}
% --------------------------
\end{table}

\end{frame}

%%% Slide 07B - Implosion Count per Number of CSI-events per year%%%
\section{Appendix}
\begin{frame}{Appendix: Number of CSIs per Year}

\begin{table}[!h]
\centering
% \caption{Temporal Distribution of CSI Active Years (Split View)}
\centering
\begin{tabular}[t]{crrcrr}
\toprule
\multicolumn{3}{c}{} & \multicolumn{3}{c}{} \\
\cmidrule(l{3pt}r{3pt}){1-3} \cmidrule(l{3pt}r{3pt}){4-6}
Year & Events & \% & Year & Events & \%\\
\midrule
\cellcolor{gray!10}{1998} & \cellcolor{gray!10}{83} & \cellcolor{gray!10}{3.71} & \cellcolor{gray!10}{2011} & \cellcolor{gray!10}{36} & \cellcolor{gray!10}{1.61}\\
1999 & 56 & 2.50 & 2012 & 80 & 3.58\\
\cellcolor{gray!10}{2000} & \cellcolor{gray!10}{124} & \cellcolor{gray!10}{5.55} & \cellcolor{gray!10}{2013} & \cellcolor{gray!10}{86} & \cellcolor{gray!10}{3.85}\\
2001 & 16 & 0.72 & 2014 & 81 & 3.62\\
\cellcolor{gray!10}{2002} & \cellcolor{gray!10}{57} & \cellcolor{gray!10}{2.55} & \cellcolor{gray!10}{2015} & \cellcolor{gray!10}{95} & \cellcolor{gray!10}{4.25}\\
\addlinespace
2003 & 44 & 1.97 & 2016 & 134 & 5.99\\
\cellcolor{gray!10}{2004} & \cellcolor{gray!10}{22} & \cellcolor{gray!10}{0.98} & \cellcolor{gray!10}{2017} & \cellcolor{gray!10}{155} & \cellcolor{gray!10}{6.93}\\
2005 & 60 & 2.68 & 2018 & 56 & 2.50\\
\cellcolor{gray!10}{2006} & \cellcolor{gray!10}{69} & \cellcolor{gray!10}{3.09} & \cellcolor{gray!10}{2019} & \cellcolor{gray!10}{185} & \cellcolor{gray!10}{8.27}\\
2007 & 68 & 3.04 & 2020 & 182 & 8.14\\
\addlinespace
\cellcolor{gray!10}{2008} & \cellcolor{gray!10}{55} & \cellcolor{gray!10}{2.46} & \cellcolor{gray!10}{2021} & \cellcolor{gray!10}{211} & \cellcolor{gray!10}{9.44}\\
2009 & 114 & 5.10 & 2022 & 97 & 4.34\\
\cellcolor{gray!10}{2010} & \cellcolor{gray!10}{70} & \cellcolor{gray!10}{3.13} & \cellcolor{gray!10}{NA} & \cellcolor{gray!10}{NA} & \cellcolor{gray!10}{NA}\\
\bottomrule
\end{tabular}
% --------------------------
\captionsetup{font=tiny}
\caption{Number of CSI events per year.}
% --------------------------
\end{table}

\end{frame}


%==== 00 - OTHER ==============================================================%

\begin{frame}{Bibliography}

\bibliographystyle{plainnat}
\bibliography{references}

\end{frame}

%==== END =====================================================================%

\end{document}
