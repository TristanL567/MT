%==== KNITR ===================================================================%



%==== START ===================================================================%

\documentclass{beamer}\usepackage[]{graphicx}\usepackage[]{xcolor}
% maxwidth is the original width if it is less than linewidth
% otherwise use linewidth (to make sure the graphics do not exceed the margin)
\makeatletter
\def\maxwidth{ %
  \ifdim\Gin@nat@width>\linewidth
    \linewidth
  \else
    \Gin@nat@width
  \fi
}
\makeatother

\definecolor{fgcolor}{rgb}{0.345, 0.345, 0.345}
\newcommand{\hlnum}[1]{\textcolor[rgb]{0.686,0.059,0.569}{#1}}%
\newcommand{\hlstr}[1]{\textcolor[rgb]{0.192,0.494,0.8}{#1}}%
\newcommand{\hlcom}[1]{\textcolor[rgb]{0.678,0.584,0.686}{\textit{#1}}}%
\newcommand{\hlopt}[1]{\textcolor[rgb]{0,0,0}{#1}}%
\newcommand{\hlstd}[1]{\textcolor[rgb]{0.345,0.345,0.345}{#1}}%
\newcommand{\hlkwa}[1]{\textcolor[rgb]{0.161,0.373,0.58}{\textbf{#1}}}%
\newcommand{\hlkwb}[1]{\textcolor[rgb]{0.69,0.353,0.396}{#1}}%
\newcommand{\hlkwc}[1]{\textcolor[rgb]{0.333,0.667,0.333}{#1}}%
\newcommand{\hlkwd}[1]{\textcolor[rgb]{0.737,0.353,0.396}{\textbf{#1}}}%
\let\hlipl\hlkwb

\usepackage{framed}
\makeatletter
\newenvironment{kframe}{%
 \def\at@end@of@kframe{}%
 \ifinner\ifhmode%
  \def\at@end@of@kframe{\end{minipage}}%
  \begin{minipage}{\columnwidth}%
 \fi\fi%
 \def\FrameCommand##1{\hskip\@totalleftmargin \hskip-\fboxsep
 \colorbox{shadecolor}{##1}\hskip-\fboxsep
     % There is no \\@totalrightmargin, so:
     \hskip-\linewidth \hskip-\@totalleftmargin \hskip\columnwidth}%
 \MakeFramed {\advance\hsize-\width
   \@totalleftmargin\z@ \linewidth\hsize
   \@setminipage}}%
 {\par\unskip\endMakeFramed%
 \at@end@of@kframe}
\makeatother

\definecolor{shadecolor}{rgb}{.97, .97, .97}
\definecolor{messagecolor}{rgb}{0, 0, 0}
\definecolor{warningcolor}{rgb}{1, 0, 1}
\definecolor{errorcolor}{rgb}{1, 0, 0}
\newenvironment{knitrout}{}{} % an empty environment to be redefined in TeX

\usepackage{alltt}

% --- THEME SETUP ---
% I replaced the custom 'ldv' theme with 'Madrid' so this compiles immediately.
% If you have the ldv files locally, uncomment the line below and remove 'Madrid'.
% \usetheme{ldv} 
\usetheme{Madrid} 
\usecolortheme{beaver} % A nice color variation

% --- PACKAGES ---
\usepackage[utf8]{inputenc}
\usepackage{amsmath,amssymb}
\usepackage{graphicx}
\usepackage{booktabs} % For professional tables
\usepackage{hyperref}

% --- METADATA ---
\title[Thesis Proposal]{The Agony and the...}
\subtitle{Master Thesis Proposal}
\author{Tristan Leiter}
\institute[WU]{Vienna University of Business and Economics}
\date{\today}

%==== Beginning of the document ===============================================%

\IfFileExists{upquote.sty}{\usepackage{upquote}}{}
\begin{document}

%==== Title Page ==============================================================%

% 1. Title Page
\begin{frame}
  \titlepage
\end{frame}

% 2. Outline
\begin{frame}{Agenda}
  \tableofcontents
\end{frame}

%==== 01 - Problem Statement ==================================================%

\section{Problem Statement}

\begin{frame}{Motivation}
  \begin{itemize}
    \item \textbf{Context:} Modern distributed systems generate massive logs.
    \item \textbf{Pain Point:} Manual analysis is error-prone and slow.
    \item \textbf{Goal:} Develop a hybrid deep-learning model for real-time detection.
  \end{itemize}
  
  \vspace{1em}
  
  \begin{alertblock}{Research Gap}
    Existing solutions struggle with high-dimensional, unlabeled data streams.
  \end{alertblock}
\end{frame}

%==== 02 - Methodology ========================================================%

\section{Methodology}

\begin{frame}{Proposed Mathematical Model}
  We assume the state transition follows a Markov process. The posterior probability is updated as follows:
  
  \begin{equation}
    p(\mathbf{x}_{k}|\mathbf{Z}_{k}) = \frac{p(\mathbf{z}_{k}|\mathbf{x}_{k})p(\mathbf{x}_{k}|\mathbf{Z}_{k-1})}{\int \! p(\mathbf{z}_{k}|\mathbf{x}_{k})p(\mathbf{x}_{k}|\mathbf{Z}_{k-1})\,d\mathbf{x}_{k}}
  \end{equation}
  
  Where:
  \begin{itemize}
    \item $\mathbf{x}_{k}$ is the system state at time $k$
    \item $\mathbf{Z}_{k}$ is the observation history
  \end{itemize}
\end{frame}

%==== 03 - Data ===============================================================%

\section{Preliminary Data}

% [fragile] is MANDATORY when using R code blocks in Beamer
\begin{frame}[fragile]{Preliminary Data Analysis (R Integration)}
  Using R directly in slides ensures plots are always up-to-date with your data.
  
\begin{knitrout}\footnotesize
\definecolor{shadecolor}{rgb}{0.969, 0.969, 0.969}\color{fgcolor}\begin{kframe}
\begin{alltt}
\hlcom{# We can generate plots dynamically}
\hlkwd{par}\hlstd{(}\hlkwc{mar}\hlstd{=}\hlkwd{c}\hlstd{(}\hlnum{4}\hlstd{,}\hlnum{4}\hlstd{,}\hlnum{1}\hlstd{,}\hlnum{1}\hlstd{))}
\hlkwd{plot}\hlstd{(iris}\hlopt{$}\hlstd{Sepal.Length, iris}\hlopt{$}\hlstd{Petal.Length,}
     \hlkwc{col}\hlstd{=iris}\hlopt{$}\hlstd{Species,} \hlkwc{pch}\hlstd{=}\hlnum{19}\hlstd{,}
     \hlkwc{main}\hlstd{=}\hlstr{"Cluster Separation Feasibility"}\hlstd{)}
\hlkwd{legend}\hlstd{(}\hlstr{"topleft"}\hlstd{,} \hlkwc{legend}\hlstd{=}\hlkwd{levels}\hlstd{(iris}\hlopt{$}\hlstd{Species),} \hlkwc{col}\hlstd{=}\hlnum{1}\hlopt{:}\hlnum{3}\hlstd{,} \hlkwc{pch}\hlstd{=}\hlnum{19}\hlstd{,} \hlkwc{cex}\hlstd{=}\hlnum{0.8}\hlstd{)}
\end{alltt}
\end{kframe}

{\centering \includegraphics[width=\maxwidth]{figure/summary_plot-1} 

}


\end{knitrout}
\end{frame}

% --- SECTION 4: TIMELINE ---
\section{Plan}

\begin{frame}{Project Timeline}
  \begin{table}[]
    \centering
    \begin{tabular}{ll}
      \toprule
      \textbf{Phase} & \textbf{Deadline} \\
      \midrule
      Literature Review & Month 1 \\
      Data Collection & Month 2 \\
      Implementation (Prototype) & Month 3-4 \\
      Evaluation \& Tuning & Month 5 \\
      Writing Thesis & Month 6 \\
      \bottomrule
    \end{tabular}
    \caption{Estimated milestones for the 6-month period.}
  \end{table}
\end{frame}

%==== 04 - Conclusion =========================================================%

\begin{frame}{Next Steps}
  \centering
  \Large \textbf{Thank you for your attention.}
  
  \vspace{1cm}
  
  \normalsize
  Questions?
\end{frame}

\end{document}
